\documentclass{beamer}
\usetheme{Berlin}
\usepackage[latin1]{inputenc}
\usepackage{alltt}
\title{Node.js}
\author{Francisco Ar\'evalo \\ \texttt{farevalod@gmail.com}}
\date{\today}
\begin{document}
\begin{frame}
\titlepage
\end{frame}
\begin{frame}
	\frametitle{Contenidos}
	\tableofcontents
\end{frame}
\section{Setup}
\subsection{Download}
\begin{frame}
\frametitle{Download}
\texttt{http://nodejs.org/dist/node-v0.4.9.tar.gz}
\end{frame}
\subsection{Install}
\begin{frame}
\frametitle{Install}
\texttt{./configure \&\& ./make \&\& (sudo) ./make install}
\end{frame}
\section{Servers}
\begin{frame}
\frametitle{VM}
\begin{itemize}
\item V8 es una m\'aquina virtual de Javascript.\\
\item node.js es una serie de librer\'ias que facilitan tareas de red.
\end{itemize}
\end{frame}
\begin{frame}
\frametitle{Never Sleeps}
\begin{itemize}
\item Los servidores en node.js nunca duermen... A lo m\'as, idlean.\\
\item Cuando el event loop est\'a vac\'io, node no ocupa el procesador!
\end{itemize}
\end{frame}
\begin{frame}
\frametitle{Siempre puede m\'as}
\begin{itemize}
\item Como funciona usando callbacks, cuando recibe un request, hace la llamada.\\
\item Siempre puede manejar una conecci\'on adicional, el l\'imite es el que impone el sistema.
\end{itemize}
\end{frame}
\subsection{Servidor http}
\begin{frame}
\frametitle{Simple servidor http}
En node.js, un servidor http se puede implementar en 5 lineas:
\begin{alltt}
var http = require('http');\\
http.createServer(function (req, res) {\\
  res.writeHead(200, {'Content-Type': 'text/plain'});\\
  res.end('Hello World');}).listen(1337, "127.0.0.1");\\
console.log('Server running at http://127.0.0.1:1337/');\\
\end{alltt}
Y listo!
\end{frame}
\subsection{echo server}
\begin{frame}
\frametitle{echo server}
Un servidor de echo en 4 lineas:
\begin{alltt}
var net = require('net');\\
var server = net.createServer(function (socket) {\\
  socket.write("Echo server");\\
  socket.pipe(socket);
}).listen(1337, "127.0.0.1");\\
\end{alltt}
Y ya acepta conecciones de los usuarios.
\end{frame}
\section{Why Node?}
\begin{frame}
\frametitle{Porque node}
\begin{itemize}
\item El servidor puede funcionar en el puerto 80, sirviendo http... o ser accesible por telnet via 8000... o recibir emails...\\
\item La VM ofrece suficiente abstracci\'on para tener todas las libertades.
\item Toda la API es en base a callbacks, asi que se pueden customizar los retornos de todas las funciones que implementa por default.\\
\item En node, el ciclo de eventos lleva la direcci\'on de lo que se ejecuta, y el proceso nunca duerme!\\
\item Las fuentes de informaci\'on pueden ser asincronas, poco fiables, bases de datos, c\'alculos... node renderea a medida que puede.\\
\end{itemize}
\end{frame}
\section{Una prueba}
\begin{frame}
\subsection{Objetivos}
\frametitle{Objetivos:}
\begin{itemize}
\item Montar un server con un protocolo arbitrario.
\item Aprovechar API para trabajar con la red.
\item Usar js para guardar y manejar objetos.
\item Programar comportamiento nuevo.
\item Reutilizar scripts existentes.
\end{itemize}
\end{frame}
\begin{frame}
\frametitle{Servidor de prueba:}
\texttt{nc 4ws.cl 8000}\\
o\\
\texttt{telnet 4ws.cl 8000}
\end{frame}
\begin{frame}
\subsection{Implementaci\'on}
\frametitle{Varios clientes conectados}
\texttt{
sockets = [];\\
sockets.push(socket);//lo agrega al array...\\
socket.on('data',function(d){\\
for(var i = 0;i < sockets.sizeOf;i++)\\
socket[i].write(d);});\\
// + codigo teardown...
}
\end{frame}
\begin{frame}
\frametitle{parser}
\texttt{
parse = function(d,socket,sockets,chars){\\
	msg = d.toString('utf8', 0, d.length - 1);\\
	if(msg.match(/name/)){\\
	re = /name/;\\
	target = re.exec(msg);}}
}
\\
Parsea input del usuario en busca de comandos o chat.
Usa un regexp para seleccionar cada comando y sus argumentos.
\end{frame}
\section{Problemas}
\begin{frame}
\frametitle{Problemas...}
\begin{itemize}
\subsection{Single Core}
\item No es problema, seg\'un el creador- a ese nivel ya se deber\'ia optimizar para correr en m\'aquinas paralelas (y para los procs est\'a el scheduler del sistema)
\subsection{Inmadurez}
\item C\'odigo muy inmaduro
Todav\'ia no existe una cantidad enorme de librer\'ias, para conectarse a dbs, drivers, etc.
\subsection{Stack traces}
\item Stack traces
Como se ocupa arquitectura de eventos, los stacks de eventos no son como los procedurales, sino que dan muy poco contexto acerca de la funci\'on.
\end{itemize}
\end{frame}
\begin{frame}
\frametitle{Links y contacto}
\begin{itemize}
\item \texttt{https://github.com/farevalod/nodemud}
\item \texttt{http://nodejs.org/dist/node-v0.4.9.tar.gz}
\item \texttt{farevalod@gmail.com}
\end{document}
